\chapter{The Time Value of Money}
\section{Interest Rate}
denoted $r$. Three Explanation:
\begin{description}
\item[Rate of Return] The minimum rate of return an investor must receive in order to accept the
investment.
\item[Discount Rate] Discount the future value to find its value today.
\item[opportunity cost] The value that
investors forgo by choosing a particular course of action.
\end{description}
$r$ $=$ Real risk-free interest rate $+$ Inflation premium $+$ Default risk premium $+$
Liquidity premium $+$ Maturity premium\par

The \textbf{real risk-free interest rate} is the single-period interest rate for a completely risk-
free security if no inflation were expected. In economic theory, the real risk-free rate
reflects the time preferences of individuals for current versus future real consumption.

The \textbf{inflation premium} compensates investors for expected inflation and reflects the
average inflation rate expected over the maturity of the debt.The sum of the real risk-free interest rate and the inflation premium is the
\textbf{nominal risk-free interest rate}. Many countries have governmental short-term debt
whose interest rate can be considered to represent the nominal risk-free interest rate in
that country.

The \textbf{default risk premium} compensates investors for the possibility that the borrower
will fail to make a promised payment at the contracted time and in the contracted amount.

The \textbf{liquidity premium} compensates investors for the risk of loss relative to an
investment’s fair value if the investment needs to be converted to cash quickly.

The \textbf{maturity premium} compensates investors for the increased sensitivity of the market
value of debt to a change in market interest rates as maturity is extended, in general.

\section{The Time Value of Money}
\textbf{simple interest, Principal, compounding}
\begin{equation}
    FV_N = PV(1 + r)^N
\end{equation}
